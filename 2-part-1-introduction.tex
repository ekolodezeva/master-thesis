\subsection{Актуальность и обоснование выбора темы}
\par
Данных ПГИ и мониторинга скважин встроенными оптоволоконными системами становится все больше, и зачастую на решение поставленных задач не хватает рук экспертов и времени и ресурсов для численного моделирования; ответ нужен «в реальном времени». Требуется разработать надежные системы автоматизированной обработки – как минимум для выполнения основной части рутинной работы, производимой зачастую вручную; и как оптимум – для подготовки вариантов интерпретации данных и рекомендации решений для выработки экспертом окончательных заключений.
\par
Методы машинного обучения могут оказать серьезную помощь в построении таких систем. Каждая скважина уникальна, и по данным невозможно однозначно описать происходящие в скважине процессы. Таким образом, возможность применения методов машинного обучения с учителем сильно ограничена, поскольку недостаточно хорошо размеченных датасетов. Поэтому в данном исследовании мы сконцентрировались на методах обучения без учителя.
\par
В данной части работы обсуждаются три задачи интерпретации данных ПГИ. Задачи связаны между собой возрастающей необходимостью уметь автоматически локализовать и классифицировать зоны притока флюидов по глубине скважины – то есть определять, на каких интервалах глубины скважины наиболее вероятна зона притока флюида и определить тип флюида – вода, нефть, газ или смесь флюидов.
