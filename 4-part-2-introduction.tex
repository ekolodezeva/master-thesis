\subsection{Актуальность и обоснование выбора темы}
\par
Общепринятые методы ПГИ позволяют получать высококачественные данные для вертикальных и околовертикальных скважин \cite{horizonal_1}.
\par
Специально для низкодебитных скважин в Шлюмберже разработан еще один прибор PLT-0.6.3, работа которого основана на явлении термоанемометрии \cite{horizonal_1},\cite{horizonal_2}. Интерпретация получаемых им данных проводится с целью определить фазовый состав, распределение и скорости флюидов по длине скважины. Разработаны процессы для калибровки и интерпретации данных термоанемометрии, результаты которых хорошо согласуются с результатами совместной интерпретации данных расходомера, влагомера и других ПГИ данных. Добавление PLT-0.6.3 в сборку приборов повышает качество интерпретации и увеличивает диапазон применимости сборки.
\par
Невозможность получения исчерпывающей информации о потоке флюида в скважине приводит к неоднозначности результатов интерпретации. Для построения наиболее правдоподобной интерпретации, причем с оценкой неопределенности, необходимо иметь различные варианты предсказаний фазового состава флюида на основе имеющейся информации. В данной работе предлагается дополняющий методы \cite{horizonal_1},\cite{horizonal_2} новый подход определения фазового состава флюида в скважине на основе измерений датчика термоанемометра. 
\par
Рассматривается горизонтальная газо-нефтяная скважина с наличием воды. Используемые для исследования данные получены с помощью каротажной сборки приборов, которая включает в себя базовый модуль PLT-9.2 (измеряемые величины: температура, давление, гамма-каротаж, влагомер, резистивиметр, акустический шумомер) и экспериментальный прототип распределенного термоанемометра PLT-0.6.3. Из временных рядов всех данных ПГИ для анализа мы используем данные температуры и термоанемометрии. При этом данные температуры предварительно корректируем для устранения тепловой инерции датчика.