\subsection{Зачем нужны промыслово-геофизические исследования}
\par
Промыслово-геофизические мероприятия проводятся для оптимизации процесса разработки месторождения с целью увеличения добычи нефти. ПГИ могут проводиться как при работающей скважине, так и при остановленной. Иногда для сбора достаточного для анализа количества данных проводится несколько исследований на различных этапах работы скважины.
\par
Одна из основных задач ПГИ - локализация зон притока флюида в скважину. Как правило, при этом решается еще и задача определения типа флюида - вода, нефть, газ или смесь флюидов - и определение характеристик флюида - плотность и скорость.
\par
Одно из наиболее эффективных мероприятий для увеличения добычи углеводородов - проведение многостадийного гидроразрыва пласта \cite{fracking}. В скважину под большим давлением (до 100МПа) закачивается жидкость разрыва высокой вязкости (на водной или нефтяной основе). Под влиянием давления жидкости разрыва пласт расщепляется. После этого в скважину закачивается жидкость-песконоситель при давлении, позволяющем удержать скважину в открытом состоянии. Жидкость-песконоситель - более вязкая, чем жидкость разрыва, и содержит проппант - песок или другие гранулы диаметром обычно порядка 0.5-1.2мм. Шарики проппанта затекают в образовавшиеся трещины. При откачке жидкости давление снижается, но образовавшиеся трещины не схлопываются полностью из-за наличия в них шариков проппанта, при этом часть проппанта разрушается, а часть вдавливается в породу. Раскрытые трещины позволяют значительно увеличить прирост нефти.

ПГИ могут проводиться при проведении многостадийного гидроразрыва пласта \cite{191562}. Анализ данных ПГИ позволяет оценить качество полученных трещин и локализовать зоны притоков в скважине.
\par
Другая часто решаемая задача - поиск неисправностей в структуре скважины, то есть нежелательных трещин, или утечек \cite{204557}.

\subsection{Оборудование для каротажа}
\par
В ходе ПГИ в скважину опускается каротажная сборка приборов, которая проводится по всей глубине скважины и непрерывно пишет данные. Приборы могут состоять из одного или нескольких датчиков, измеряющих различные физические величины.

\subsubsection{Стандартные датчики}
\par
Часто используемый для ПГИ прибор PLT-9.2 \cite{92} содержит в себе датчики температуры, давления, влагомер и расходомер. Для привязки показаний прибора к траектории скважины используется локатор муфт и датчик уровня естественного гамма-излучения \cite{169255}.

\subsubsection{Акустическая шумометрия}
\par
В рамках измерения спектральный шумомер записывает акустические сигналы в скважине в широком диапазоне частот с хорошим разрешением: анализируемые в данной работе данные пассивной акустической шумометрии были получены приборами с диапазоном измерений от 8Гц до 60кГц и от 8Гц до 100кГц, записанными в 512 или 1024 частотных каналов с высокой чувствительностью. Каждой точке траектории, в которой производились измерения, сопоставляется запись акустического шума порядка 0.1сек с частотой дискретизации 200кГц, которая для анализа обычно преобразуется в вектор, компоненты которого соответствуют интенсивности сигнала в определенном диапазоне частот. Запись показаний прибора может проводиться как при движении прибора, так и на остановках (например, каждые 2 метра) - чтобы избежать дорожного шума, который может скрыть низкочастотные сигналы малой интенсивности.
\par
Акустический шум появляется при турбулентном движении флюида по стволу скважины.. Согласно \cite{lighthill}, \cite{191562}, мощность акустического излучения W при таком движении зависит от параметров трубы и флюида:
\begin{equation}
\label{eq:power_law}
    W\sim\frac{\rho U^8 d^2}{a_0^5},
\end{equation}
Где $\rho$ – средняя по сечению трубы плотность флюида, $a_0$ – скорость звука во флюиде, $U$ – скорость потока и $d$ – диаметр трубы, который задает характерный масштаб турбулентности.
\par
Так как мощность зависит от скорости в восьмой степени, шумометрия - очень чувствительный способ измерения скорости потока. Это позволяет достичь хороших результатов в случае однофазного течения; в случае смеси флюидов меняются и другие компоненты формулы \eqref{eq:power_law} - плотность флюида и скорость звука во флюиде, поэтому показания шумометрии нужно рассматривать в комбинации с остальными показаниями датчиков.
\par
Показано \cite{pa_good}, \cite{187909}, что сочетание пассивной акустической шумометрии с классическими приборами геофизических исследований позволяет улучшить результаты анализа данных ПГИ. 
\par
В частности, одно из достоинств спектральной шумометрии заключается в том, что с ее помощью можно выделить и дифференцировать заколонные перетоки - то есть потоки флюида между обсадной колонной и породой, которые невозможно определить с помощью, например, расходомера. Выявление заколонных перетоков важно для проверки технического состояния скважины, определения расположения продуктивных пластов, оценки эффективности гидроразрыва и т.п.
\par
Также чувствительность современных приборов спектральной шумометрии позволяет уловить притоки со скоростями ниже чувствительности обычных расходомеров, которые при таких скоростях просто не раскручиваются. Поэтому акустическая шумометрия особенно важна в горизонтальных и около-горизонтальных малодебитных скважинах.
\par
Последнее время актуальными становятся измерения акустической шумометрии и температуры с использованием оптоволоконных кабелей – технологии DTS (distributed temperature sensing) и DAS (distributed acoustic sensing) \cite{161712}.

\subsection{Анализ данных ПГИ}
\par
Анализ полученных в результате каротажных исследований данных может проводиться как в ручном режиме, так и с использованием написанных экспертами программ. Невозможность непосредственно измерить такие важные показатели, как пористость и плотность пластов, делают очень сложной задачу численного моделирования скважины для решения задачи с высокой точностью, поэтому обычно используется ручной анализ и/или численное моделирование \cite{196955}.
\par
Рассмотрим для примера задачу локализации зон притока с использованием данных спектральной акустической шумометрии.
\par
При наличии данных пассивной акустической шумометрии «ручной» способ анализа данных обычно предполагает построение спектрограммы в смысле изменения спектра сигнала при продвижении вдоль траектории скважины. Изменения спектра сигнала могут указывать на наличие зон притока.
\par
В \cite{187909} описаны способы локализации зон притока в малодебитных горизонтальных скважинах с использованием спектральной акустической шумометрии. Описаны некоторые характеристики спектрограмм, позволяющие выделить события: так, поток флюида через трубу большого диаметра генерирует низкочастотный шум, а через трубу меньшего диаметра - высокочастотный \cite{162081}; также поток через пористую среду генерирует высокочастотный шум, а течение через естественную или искусственно созданную скважину - низкочастотный. Для избавления от статистически не значимого шума, вызванного движением прибора или фоновым шумом скважины, используются различные техники фильтрации шума, например, на основе вейвлет-преобразования \cite{162081}. В целом притоки углеводородов можно выделить по широкому спектру производимого акустического шума.

\subsection{Решаемые в исследовании задачи}
\par
В данной работе рассмотрены две задачи: локализация зон притока в скважину и определение фазового состава флюида по данным термоанемометрии. 
\par
\textit{Задача локализации зон притока} решается на основе задачи выявления событийных участков в скважине. Событийными участками могут быть притоки воды, нефти или газа, а также изменения показаний приборов, связанные с вариацией траектории скважины. Это могут быть удары каротажной сборки приборов о стенки скважин, изменения показаний из-за стыка труб, или, в случае горизонтальных скважин, индикаторы стратификации жидкости не в местах наличия притоков - в частности, могут существовать места скопления воды на вогнутых по отношению к горизонту участках траектории и газовые облака на выпуклых участках. Так как задача разделения участков на зоны притока и зоны стратификации жидкостей, не связанных с притоком, требует дальнейших кропотливых исследований, в том числе учета мнения экспертов, в данной работе она не рассмотрена. 
\par
Тем не менее, как будет показано далее, предложенный для решения задачи выявления событийных участков подход может быть доработан до семейства алгоритмов, решающих больший диапазон задач анализа данных ПГИ, в том числе и задачу локализации зон притока. В частности, предложенный подход может помочь в решении любой проблемы, которую можно свести к задаче классификации - причем как бинарной классификации, так и задачи классификации на несколько классов\cite{ws_multiclass}. В качестве задачи классификации на несколько классов можно привести задачу локализации зон притока с указанием типа притока - вода, нефть или газ; эта же задача может решаться с применением трех моделей бинарной классификации. 
\par
\textit{Задача определения фазового состава флюида вдоль ствола скважины по данным термоанемометрии} решается с помощью анализа геометрической формы циклов термоанемометрического датчика. Форма зависит от типа флюида, где находится датчик во время цикла (12 сек). Показано, что можно использовать два признака геометрии цикла, вычисляемых на основе значений температуры – как фоновой температуры вдоль оси скважины, так и показаний специального распределенного прибора термоанемометрии, – в качестве координат пространства для кластерного анализа. В этих координатах удается разделить точки-циклы на три достаточно отчетливых кластера, соответствующих воде, нефти и газу. Для учета неопределенности разделение проводится с помощью модели смеси гауссовских распределений, позволяющей приписать каждой точке-циклу вероятность принадлежности к тому или иному кластеру. Это дает возможность не только определить для каждой точки наиболее вероятный соответствующий ей тип флюида, но и оценить фазовый состав при предполагаемом наличии смеси флюидов.

