\par
В работе описаны алгоритмы решения двух типов задач промыслово-геофизических исследований: первый охватывает поиск событийных участков и локализацию зон притока; второй –  оценку распределения фазового состава флюидов по длине скважины. Все исследования проводились на полевых данных реальных скважин.
\par
Использованы алгоритмы машинного обучения без учителя – модель слабого контроля и кластеризация смеси гауссовых распределений; разработаны прототипы программ на языке Python. Во всех задачах в той или иной мере применялись методы цифровой обработки сигналов. При этом уже на этапе цифровой обработки сигналов при подборе параметров фильтрации и т.п. может возникнуть необходимость в машинном обучении без учителя;  для этого также возможно использование модели слабого контроля, описанной в первой части.
\par
В целом хочется отметить широкий потенциал модели слабого контроля как модели-советника по принятию решений для эксперта. Представляется, что широкий класс задач анализа данных ПГИ можно формулировать таким образом, что в качестве основы алгоритма будет выступать классификация – на два класса, на несколько или даже с присвоением нескольких меток одному объекту. При наличии достаточного количества экспертных правил (разметочных функций), позволяющих решить задачу в такой формулировке, модель слабого контроля должна выдавать надежные результаты, обоснованные теорией вероятностного машинного обучения.
\par
Важно также отметить следующее. Каждая скважина уникальна, поэтому проверка разработанных алгоритмов требует длительной и кропотливой работы по анализу их применения для все новых и новых датасетов (скважин). Более того, эта проверка неизбежно влечет за собой улучшение первоначальных алгоритмов, т.е. дальнейшую разработку и модификацию предложенных подходов.
\newline
\par
Работа выполнена при финансовой поддержке Московского научно-исследовательского центра «Шлюмберже».
